\section{Виды структур и как их получить}
Основные форматы структур больших биомолекул~-- PDB и mmCIF. Оба формата текстовые, с определенным 
синтаксисом\footnote{Документация: \url{https://www.wwpdb.org/documentation/file-format}}. 
Формат PDB считается устаревшим, но все еще широко используется. Если вы начинаете новый проект, 
связанный со структурами биомолекул, лучше использовать mmCIF там, где это возможно, так как этот
формат устойчивее к ошибкам и позволяет хранить больше метаданных.

Малые молекулы, например, метаболиты, коферменты и лекарственные препараты тоже можно закодировать в PDB, но еще довольно часто встречаются форматы: \href{https://reference.wolfram.com/language/ref/format/MOL2.html}{MOL2}, \href{http://libatoms.github.io/QUIP/io.html#extendedxyz}{XYZ} и однострочники (\href{http://opensmiles.org/opensmiles.html}{SMILES} и \href{https://pubs.acs.org/doi/pdfplus/10.1021/ci960109j}{SYBYL Line Notation})

Чаще всего, структура, с которой вы работаете, уже содержится в какой-нибудь базе данных и иметь файл с ней необязательно, достаточно знать идентификатор (PDB~ID, например)~-- Химера сама загрузит всю необходимую информацию с сервера БД.

\paragraph{Распространенные базы данных структур биомолекул:}
\begin{itemize}
    \item PDB (\href{https://rcsb.org/pdb}{RCSB}, \href{https://www.ebi.ac.uk/pdbe}{PDBe}, \href{https://pdbj.org}{PDBj})~-- структуры белков;
    \item \href{http://ndbserver.rutgers.edu}{NDB}~-- структуры нуклеиновых кислот;
    \item \href{http://scop2.mrc-lmb.cam.ac.uk}{SCOP2} и \href{http://www.cathdb.info}{CATH}~-- эволюция белков, структурные семейства, домены, ук\-лад\-ка белков;
    \item \href{https://www.ebi.ac.uk/pdbe/emdb/}{EMDB}~-- электронная микроскопия.
\end{itemize}

\paragraph{Базы данных малых молекул}
\begin{itemize}
    \item \href{https://pubchem.ncbi.nlm.nih.gov/}{PubChem}~-- база данных химических соединений от NCBI;
    \item \href{https://www.ebi.ac.uk/chembl/}{ChEMBL} и \href{https://www.ebi.ac.uk/chebi/}{ChEBI}~-- база данных и онтология биомолекул от EBI;
    \item \href{http://zinc15.docking.org}{ZINC}~-- база данных коммерчески доступных соединений (оптимизирована для виртуального скринига);
    \item \href{https://www.ccdc.cam.ac.uk}{CCDC}~-- кембриджская кристаллографическая база данных;
\end{itemize}

\subsection*{Задание~1}
\begin{enumerate}
    \item Запустите Химеру.
    
    \item \textit{Загрузим структуру с сервера PDB:}\\
        В меню\quad\texttt{File~> Fetch by ID...}\quad выберите пункт\quad\texttt{PDB (mmCIF)}, введите \texttt{4MNF}
        и нажмите\quad\texttt{Fetch}.\\
        \textit{Должна открыться структура белка BRAF дикого типа.}
    
    \item \textit{Теперь откроем обычный PDB-файл:}\\
        В меню\quad\texttt{File~> Open...}\quad откройте файл\quad\texttt{braf\_v600e.pdb}.\\
        \textit{К уже открытой структуре белка BRAF WT добавилась структура мутантной формы BRAF V600E.} 
        % PDB ID: 1UWH
        
    \item Выведите список моделей\quad\texttt{Tools~> General Controls~> Model Panel}.\\
        Сколько моделей открыто? Сколько из них активно? Попробуйте скрыть и снова показать модель.
\end{enumerate}
