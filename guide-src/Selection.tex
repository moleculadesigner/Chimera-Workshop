\section{Выделение и группы атомов}
Команды, которые мы изучили выше влияют на отображение молекулы в целом. Если же мы хотим поработать только с определенными атомами или остатками, не затрагивая все остальное, то нам нужно будет использовать инструмент <<выделение>> (\texttt{select}).

Выделить интересующие вас атомы и связи можно с помощью мышки, зажав \texttt{Ctrl}. Если выделить за один раз все не получается, выделение можно расширить, зажав \texttt{Shift + Ctrl}.

\subsection*{Задание~4}
\begin{enumerate}
    \item Выделите N-концевые остатки полипептидных цепей (Asp 449 для дикого типа и Asp 447 для V600E).
    
    \item Сделайте выделенные атомы видимыми.
    
    \item Выделите все атомы воды:\quad\texttt{Select > Residue > HOH}\quad
    
    \item Покажите их.
    
    \item Удалите выделенные атомы:\quad\texttt{Actions > Atoms/Bonds > delete}.
    
    \item \textit{Сделаем малые молекулы и ионы видимыми.}\\
    Выделите все нестандартные остатки:\\\texttt{Select > Residue > all nonstandard}
    
    \item Представьте их в виде шариков и палочек:\\\texttt{Actions > Atoms/Bonds >
    show $\rightarrow$ ... > ball \& stick}
    
    \item \textit{Оставим в выделении только лиганды.}\\
    Поменяйте режим выделения на вычитание:\\
    \texttt{Select > Selection Mode > subtract}\\
    Уберите из выделения атомы хлора:\\
    \texttt{Select > Residue > CL}
    
    \item \label{task:sel} Сохраните лиганды в именованную группу:\\
    \texttt{Select > Name Selection ...}\quad введите имя для группы атомов.\\
    Теперь чтобы выбрать именованную группу нажмите\\\texttt{Select > Named Selections > <Имя\_Группы>}
\end{enumerate}