\section{Структурное выравнивание}
Сейчас открытые модели расположены довольно далеко друг от друга и сравнивать их не очень удобно. Чтобы наложить похожие структуры дну на другую, можно использовать инструмент MatchMaker, который проводит структурное выравнивание макромолекул.

\subsection*{Задание~5}
\begin{enumerate}
    \item \textit{Наложим структуру мутантной формы белка на структуру дикого типа.}\\
    Запустите диалоговое окно\quad\texttt{Tools~> Structure~Comparison > Match\-Maker}
    
    \item Выберите первую модель (\#0) в списке \texttt{Reference structure} и вторую (\#1) в списке 
    \texttt{Structure(s) to match}. Нажмите \texttt{Ok}. Что произошло?
    
    \item \textit{Оставим по одной полипептидной цепи от каждого белка.} \\
    Выделите все цепи с индексом <<B>>:\quad\texttt{Select~> Chain~> B~> all}.\quad
    \textit{Убедитесь, что установлен режим выделения} \texttt{Replace}.
    
    \item Удалите выделенные атомы.
    
\end{enumerate}

Какие различия в структурах BRAF дикого типа и V600E вы видите?