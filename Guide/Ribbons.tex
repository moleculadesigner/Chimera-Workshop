\section{Способы представления молекулы белка}
Во время анализа структуры биомолекулы нас могут заинтересовать разные уровни
организации: от отдельных атомов до межмолекулярных комплексов. Для каждого из уровней удобнее использовать свое представление:
для визуализации активных центров рисовать отдельные атомы, для структуры белка в целом~-- пептидный остов, для белковых
комплексов~-- поверхность доступности растворителя.

\subsection*{Задание~3}
\begin{enumerate}
    \item Выключите отображение остова и включите отображение атомов:\\\quad\texttt{Actions~> Ribbon~> hide}\quad
    $\rightarrow$\\\quad\texttt{Actions~> Atoms/Bonds~> show}.\quad Какие атомы появились? \\
    \textit{Можно сделать картинку не такой шумной:}\\\quad\texttt{Actions~> Atoms/Bonds~> wire}.
    \item Скройте атомы и верните остов.
    \item Постройте поверхность доступности растворителя\\\quad\texttt{Actions~> Surface~> show}.\quad\\ Установите прозрачность поверхности на 50\,\%\\
    \quad\texttt{Actions~> Surface~> transparency~> 50\,\%}.\quad\\Какую информацию может дать о молекуле ее поверхность доступности растворителя?
    \item Скройте поверхность\quad\texttt{Actions~> Surface~> hide}.\quad\textit{Она понадобится немного позже.}
\end{enumerate}