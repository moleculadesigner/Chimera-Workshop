\section{Водородные связи и межатомные \\ контакты}
UCSF Chimera может предсказывать водородные связи, межатомные контакты и отталкивающие взаимодействия в исследуемых структурах. Для этого служат два инструмента: \texttt{FindHBond} и \texttt{Find Clashes/Contacts}.

К сожалению, связи и контакты предсказываются только на основе геометрии типа атомов, заряды и дипольные моменты не учитываются, поэтому, данные о взаимодействиях, полученные в химере, пригодны только для первичного анализа структур.

\subsection*{Задание~6.1}
\begin{enumerate}
    \item Выделите группу лигандов\\\texttt{Select > Named Selections > <Имя\_Группы>}\\
    \textit{см. задание~4.\ref{task:sel}}
    
    \item Вызовите диалоговое окно\\\texttt{Tools~> Structure Analysis~> FindHBonds}
    
    \item Выберите опцию поиска связей только внутри моделей (радиокнопка \texttt{intra-model})
    
    \item Чтобы ограничить поиск, выберите опцию\\
    Only find H-bonds\quad\texttt{with at~least one end selected}.
    
    \item Нажмите\quad\texttt{Apply}.
    
    \item Проанализируйте водородные связи лиганд-белок для каждой модели.
\end{enumerate}

\subsection*{Задание~6.2}
\begin{enumerate}
    \item Вызовите диалоговое окно\\\texttt{Tools~> Structure Analysis~> Find Clashes/Contacts}.
    
    \item Отметьте выделенные атомы как основу для поиска (нажмите\quad\texttt{De\-sign\-ate}) и все остальные атомы модели как вероятные контакты (радиокнопка\quad\texttt{other atoms in the same model}).
    
    \item Выберите настройки для поиска контактов (кнопка\quad\texttt{contact}).
    
    \item Нажмите\quad\texttt{Apply}.\quad Что получилось?
\end{enumerate}