\section{Общие настройки внешнего вида}
По умолчанию структуры в химере рисуются на черном фоне, а иллюстрации для журнала удобнее делать на белом. Быстрее всего поменять фон можно, выбрав одну из предустановленных настроек отрисовки молекул.
\subsection*{Задание~2.1}
В меню\quad\texttt{Presets}\quad выберите пункт\quad\texttt{Publication~1 (silhouette, rounded ribbon)}.\quad Что изменилось? Попробуйте выбрать другие опции.
\subsection*{}
Внешний вид сцены можно настроить более детально:
\subsection*{Задание~2.2}
\begin{enumerate}
    \item В панели\quad\texttt{Actions~> Color~> all options...}\quad измените цвет фона (кнопка\quad\texttt{more...}\quad напротив пункта\quad\texttt{background}) на тот, который вам нравится.
    \item В панели\quad\texttt{Tools~> Viewing controls~> Effects}\quad настройте толщину обводки, тени и затемнение перспективы.
    \item Там же, во вкладке\quad\texttt{Lightning}\quad настройте освещение. Сравните режимы\quad\texttt{Ambient}\quad и\quad\texttt{Two-point}.\quad В чем разница?
    \item В панели\quad\texttt{Tools~> Depiction~> Rainbow}\quad раскрасьте каждую цепь в отдельный цвет.
    \item Скройте все атомы:\quad\texttt{Actions~> Atoms/Bonds~> hide}. Что осталось?
\end{enumerate}